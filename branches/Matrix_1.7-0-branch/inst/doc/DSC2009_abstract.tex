\documentclass[12pt,a4paper]{article}
% \usepackage{graphicx} % if you need it

\newif\iflong% they want 1/2 to 1 page at most :
\longfalse
%\longtrue
%
\title{Exploiting sparsity in model matrices}
\author{
  Douglas Bates\\
  Dept. of Statistics, University of Wisconsin - Madison, U.S.A.\\
  \iflong E-mail: \texttt{bates@r-project.org}\\ \fi
  and\\
  Martin M\"{a}chler\\
  Seminar f\"ur Statistik, ETH Zurich, Switzerland
  \iflong \\ E-mail: \texttt{maechler@r-project.org}
  \else \\[\parskip]
  E-mails: \texttt{(bates|maechler)@r-project.org}
  \fi
}

\date{} %% I.e. leave it out!
\begin{document}
\maketitle\thispagestyle{empty} %% Avoids page number
Model matrices are used to define many types of statistical models and
fitting such models continues to be a challenge because the models
become more complex and are applied to ever larger data sets.
Traditional dense matrix decomposition methods, such as the QR and
Cholesky decompositions,
\iflong perhaps augmented with accelerated BLAS or even GPU-based BLAS, \fi
continue to be effective when the model matrices
have a moderate number of columns and it is only the number of rows
that increases.  However, in many applications the number of rows
and the number of columns in the model matrix both increase for larger
data sets.  That is, we add ``nuisance parameters'' as we add
observations.  Many mixed-effects models have this property.

Storing and decomposing dense model matrices that expand in both
height and width for larger data sets can quickly swamp computing
resources.  The computational burden becomes even greater if fitting
the model requires iterative optimization of a criterion and the
decomposition must be calculated or updated during each iteration.
Exploiting sparsity in the model matrices, if present, can reduce both
the amount of storage and the amount of computing required to fit the
statistical model.  Sparse matrix decomposition methods are
particularly effective inside iterative algorithms because the
decomposition is performed in a symbolic phase,
\iflong which determines the location of the nonzeros in the result, \fi
followed by a numeric phase%
\iflong\ to determine the actual values in those positions\fi.
Frequently the
symbolic phase is the most time-consuming but it only needs to be done
once.  Another, often overlooked, aspect of the sparse Cholesky
\iflong\else (and QR) \fi
decomposition
\iflong (and, to a lesser extent, the sparse QR decomposition) \fi
is the use of fill-reducing permutations.  Depending on the model, a
reordering of the columns in the model matrix can greatly reduce the
number of nonzeros in the decomposition and hence the
\iflong amount of computing required to determine these values.
\else effort of computing them. \fi

Sparse matrix methods implemented in the Matrix package for R are
central to the methods for mixed models implemented in the lme4
package.  We will describe in detail the theory and computational
methods for linear, generalized linear and nonlinear mixed models as
implemented in lme4 and how access to the CHOLMOD library of C
functions for the sparse Cholesky decomposition allows for a
particularly efficient mixed model implementation.

One impediment to more widespread use of sparse matrix methods in R is
the inability to directly produce a sparse model matrix from the
\texttt{model.matrix} function.  In lme4 sparse model matrices are used only
for the random effects and the nature of the random effects terms
makes it feasible to have custom code to produce the sparse matrices
directly.  It may be worthwhile providing a more general mechanism
through the \texttt{model.matrix} function.

\end{document}

